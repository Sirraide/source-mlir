\documentclass[a4paper, 12pt, oneside, final]{article}
\usepackage[british]{babel}
\usepackage{csquotes}
\usepackage{unicode-math}
\usepackage[margin=2cm]{geometry}
\usepackage{xcolor}
\usepackage{color}
\usepackage{amsmath}
\usepackage{mathtools}
\usepackage{ltx-highlight-code}
\usepackage{enumitem}

\AtBeginDocument {
\catcode`⁂=0
}

\catcode`⟨=1\relax
\catcode`⟩=2\relax
%\def⟨{\begingroup}
%\def⟩{\endgroup}

\MDDefineColorStyle{Default}{Keyword}{HTML}{FF6188}
\MDDefineColorStyle{Default}{Number}{HTML}{AB9DF2}
\MDDefineColorStyle{Default}{Function}{HTML}{A9DC76}
\MDDefineColorStyle{Default}{Operator}{HTML}{FF6188}
\MDDefineColorStyle{Default}{Type}{HTML}{78DCE8}
\MDDefineColorStyle{Default}{String}{HTML}{FFD866}
\MDDefineColorStyle{Default}{Comment}{HTML}{9B8360}
\MDDefineColorStyle{Default}{Escape}{HTML}{78DCE8}
\MDSetCodeBlockLineSkip{0pt}
\MDSetHighlightExe{./ltx-highlight-code}

\definecolor{mathcolour}{HTML}{FFD87F}
\everymath{\color{mathcolour}}

\makeatletter

%%%%%%%%%%%%%%%%%%%%%%%%%%%%%%%%%%%%%%%%%%%%%%%%%%%%%%%%%%%%%%%%%%%%%%%%%%%%%%%
%% General formatting.
%%%%%%%%%%%%%%%%%%%%%%%%%%%%%%%%%%%%%%%%%%%%%%%%%%%%%%%%%%%%%%%%%%%%%%%%%%%%%%%
%% Set fonts.
\setmainfont{Minion 3}
\setmonofont{Fira Code}[Scale=.8]
\setmathfont{latinmodern-math.otf}
\setmathfont[range=\mathit]{Minion 3 Italic}

%% Set foreground and background colours.
\definecolor{bgcolour}{HTML}{1e1e1e}
\pagecolor{bgcolour}
\color{white}

%% Algorithm name.
\def\alg#1{\ensuremath{\text{\scshape #1}}}

%% Section names etc. should not be in bold.
\def\section{\@startsection{section}{1}{0mm}{-3.5ex}{3.5ex}{\normalfont\fontsize{24pt}{36pt}\selectfont}}

\def\sref[#1]{[#1]}

\def\Example{\ifvmode\else\unskip\par\fi\noindent \textbf{Example}\par}

%% Miscellaneous.
\raggedbottom
\frenchspacing

\makeatother
\begin{document}
%%%%%%%%%%%%%%%%%%%%%%%%%%%%%%%%%%%%%%%%%%%%%%%%%%%%%%%%%%%%%%%%%%%%%%%%%%%%%%%
%% Title Page.
%%%%%%%%%%%%%%%%%%%%%%%%%%%%%%%%%%%%%%%%%%%%%%%%%%%%%%%%%%%%%%%%%%%%%%%%%%%%%%%
\begingroup
    \parindent=0pt
    \centering
    \fontsize{36pt}{48pt}\selectfont Source Language Reference \par \bigskip\bigskip
    \Large \today \par
\endgroup

%%%%%%%%%%%%%%%%%%%%%%%%%%%%%%%%%%%%%%%%%%%%%%%%%%%%%%%%%%%%%%%%%%%%%%%%%%%%%%%
%% TOC.
%%%%%%%%%%%%%%%%%%%%%%%%%%%%%%%%%%%%%%%%%%%%%%%%%%%%%%%%%%%%%%%%%%%%%%%%%%%%%%%
\tableofcontents
\clearpage

%% %%%%%%%%%%%%%%%%%%%%%%%%%%%%%%%%%%%%%%%%%%%%%%%%%%%%%%%%%%%%%%%%%%%%%%%%%%%%
%%  Type System
%% %%%%%%%%%%%%%%%%%%%%%%%%%%%%%%%%%%%%%%%%%%%%%%%%%%%%%%%%%%%%%%%%%%%%%%%%%%%%
\section{Type System [types]}
\subsection{Variants [types.variants]}
\subsubsection{Definition [types.variants.def]}
Source does not have a dedicated union or variant type. Rather, struct types may contain variant clauses
introduced by the `variant` keyword. The following example shows a struct type with two variant clauses:
```[Source]
struct foo {
    i64 a;
    variant bar { i64 c; i64 d; }
    variant baz { i32 d; }
};
```

The semantics of this type is that it contains a field `a` of type `i64`, and either
\begin{itemize}
    \item a field `c` of type `i64` and a field `d` of type `i64`, or
    \item a field `d` of type `i32`.
\end{itemize}

TODO: Variant groups (e.g. ast node vs ast type)

\subsubsection{Variant Storage [types.variants.storage]}
Each variant clause is itself a structure type and may contain any number of fields or variant clauses. All
variant clauses of a structure type are stored in overlapping memory (called the \emph{variant storage}); that is,
no matter which variant an instance of such a type stores, the memory occupied by the instance is always the same.

The variant storage behaves as though it were a `u8[N]`, where `N` is the size of the largest variant clause plus
the size of the variant index if applicable (see next section), and is aligned to the largest alignment amongst all variant
clauses (or to that of the variant index if it is placed inside the storage and if its alignment is greater than that of any
of the variant clauses).

The variant storage of a type is always placed last after any non-variant fields of the type, irrespective of where the variant
clauses are declared in the type definition.

\subsubsection{Variant Index [types.variants.index]}
To determine which variant an instance of a variant type stores, a variant index is used. The variant index is a
non-negative integer that has a different value for each variant clause of the type. The actual value of the index
for any given variant clause\footnote{Excepting that of a void-variant-clause [types.variants.empty].} is
implementation-defined and can be accessed using the `::index` metaproperty
of the corresponding variant clause (e.g. `foo::bar::index` is the index of the `bar`
variant clause of the `foo` type).

The type of the variant index shall be the smallest unsigned integer type that can represent the number of variant clauses
of the type, minus one. For example, if a type has 256 variant clauses, the variant index shall be of type `u8`.
If a type has 257 variant clauses, the variant index shall be of type `u16`, etc. If there is no such type, the
program is ill-formed.

The variant index is a compiler-generated field that shall normally be placed at the beginning of the variant storage as though
each variant clause had the index as its first field. However, if the containing structure type contains a contiguous
sequence of padding bytes of sufficient size and alignment to hold the variant index, it is placed in the first such
occurrence of bytes instead; this includes padding bytes that may be added after the variant storage at the end of the type.

A user-defined variant index may be declared by declaring a variable of unsigned integer type of sufficient size to hold
the number of variant clauses of the type, minus one, and annotating it with the `[[variant\_index]]` attribute. In that case,
no variant index is generated by the compiler and the user-defined variant index is used instead. If more than one field
of the type is annotated with the `[[variant\_index]]` attribute, the program is ill-formed.

\subsubsection{Variant Count [types.variants.count]}
The number of variant clauses of a type can be accessed using the `::variant_count` metaproperty of the type
(e.g. `foo::variant_count` is the number of variant clauses of the `foo` type). It is an error
to attempt to access the value of this metaproperty on a type that is not a structure or variant type.

\subsubsection{Initialisation and Empty variant [types.variants.empty]}
The variant stored in a default-initialised instance of a variant structure type is implementation-defined. Note
that by default, a variant cannot be ‘empty’; that is, it must always store a value of one of its variant clauses.
As a special case, an explicit void-variant-clause can be created by adding a `variant void` ‘field’ to the
structure. If such a field is present, it will be the default variant stored in a default-initialised instance of
the type, and its variant index is always 0. For example:
```[Source]
struct maybe_int {
    variant int { i64 value; }
    variant void;
}
```

The semantics of this type is that it contains either a field `value` of type `i64`, or nothing at all.
A default-initialised instance of this type will always store the void variant.

Multiple instances of a void-variant-clause in the same structure are ill-formed. Note that a variant clause with no members
(e.g. `variant empty {}` ) is \textit{not} a void-variant-clause.

\subsubsection{Accessing the Variant Type and Storage [types.variants.variant]}
The type of all variant clauses can be accessed using the by a metadata expression using the `variant` keyword.
For example, `foo::variant` is conceptually equivalent to:
```[Source]
struct foo::variant {
    variant bar { i64 c; i64 d; }
    variant baz { i32 d; }
}
```

The variant storage can be accessed by means of a member access expression whose member is the `variant` keyword. For
example, the following code declares a variable storage of type `foo::variant` and assigns it the value
of the variant storage of `v`:
```[Source]
var storage = v.variant;
```

\subsubsection{Operator \texttt{is} [types.variants.is]}
The `is` operator can be used to check whether a value contains a particular variant. For example, the following
code checks whether a value `v` of type `foo` contains a variant of type `bar`:
```[Source]
if v is bar {
    /// ...
}
```

Note that the expression `⁂emph⟨expr⟩ is bar` is ill-formed if \emph{expr} is not an expression of
structure type that contains a variant clause of type `bar`.

\subsubsection{Operator \texttt{as} [types.variants.as]}
The `as` operator can be used to extract a particular variant from a value. For example, the following code
creates a variable `bar_value` that holds the variant value of the bar a variable `v` of type `foo`.
That is, the type of `bar_value` is `(i64, i64)`:
```[Source]
var bar_value = v as bar;
```

Note that the expression `⁂emph⟨expr⟩ a bar` is ill-formed if \emph{expr} is not an expression of
structure type that contains a variant clause of type `bar`.

If `bar` is not the active variant, the value of `bar_value` will be the first
`foo.bar::size` bytes of the variant storage of `v`. This means that type punning
with variants is possible.

\subsection{References [types.ref]}
Reference types are denoted by the `&` type qualifier. The value of a reference type is the address of an object; references
always point to valid objects and can never be `nil`. Unlike references in other languages, they can, however, be reassigned.
For ease of use, references are subject to several implicit conversion, depending on the context in which they are used.

\subsubsection{Implicit Dereferencing [types.ref.autoderef]}
Rvalues of reference type can be implicitly converted to lvalues of the referenced type; that is, if `T` is a type, then
an rvalue `T&` is implicitly convertible to an lvalue of type `T`. This conversion is called \emph{implicit dereferencing}
or \emph{autodeferencing}.

\subsubsection{\texttt{is} and \texttt{cond} [types.variants.cond]}
\noindent The body of an \emph{if-expression} whose condition is an \emph{is-expression} is an \textit{implicit with-expression}
bound to the value of the `bar` variant of `foo`:
```[Source]
if v is bar then print(c + d);
```

That is, the members of the `bar` variant are in scope. Furthermore, a \textit{cond-expression} can be used in a
similar manner for pattern matching:
```[Source]
cond v is {
    bar: print(a + c + d);
    baz: print(a + d);
}
```

The body of each \textit{cond-case} (excepting the \textit{else-case}) of such a \textit{cond-expression} is an
\emph{implicit with-expression} bound to the value of the corresponding variant. Moreover, the entire \textit{cond-expression} is
an \emph{implicit with-expression} bound to the value of `v`; this pulls in all regular fields of `foo` as well. As with
all \textit{cond-expression}s using `is`, the `fallthrough` keyword is not allowed.

\subsection{Type Equality [types.equality]}
Equality of types (denoted with $=$) is an equivalence relation that is defined as follows:
\begin{itemize}
\item Every type is equal to itself.
\item Two array or vector types are equal if they are of the same kind and their element types are equal and they have the same dimension.
\item Two slice, range, or \emph{pointer-like} types are equal if they are of the same kind and their element types are equal.
\item Two tuple types $\tau'$, $\tau''$ are equal if they have the same number of elements, and for their
      element types $\tau'_i$, $\tau''_i$, it holds that $\tau'_i = \tau''_i$.
\item Two function types $\phi'$, $\phi''$ are equal if
    \begin{itemize}
        \item they have the same number of parameters, and
        \item for their parameter types $\phi'_i$, $\phi''_i$, it holds that $\phi'_i = \phi''_i$, and
        \item their return types are equal, and
        \item both are variadic or neither one is.
    \end{itemize}
\item Two closure types are equal if their function types are equal.
\end{itemize}

%% %%%%%%%%%%%%%%%%%%%%%%%%%%%%%%%%%%%%%%%%%%%%%%%%%%%%%%%%%%%%%%%%%%%%%%%%%%%%
%%  Expressions
%% %%%%%%%%%%%%%%%%%%%%%%%%%%%%%%%%%%%%%%%%%%%%%%%%%%%%%%%%%%%%%%%%%%%%%%%%%%%%
\section{Expressions [expr]}
\subsection{Value Category [expr.value]}
Every expression has an associated \emph{value category} that determines whether it is an \emph{lvalue} or an \emph{rvalue}. The
value category is uniquely determined by the expression kind and the value category of its operands, after any implicit
conversions, if applicable. Note that an expression’s value category is independent of its type.

An \emph{rvalue} is a temporary object; it may, but need not, live in memory, and as such,
it is invalid to bind a reference to an rvalue. An \emph{lvalue} is an object that has a memory address, and references
can bind to lvalues and lvalues only. Note that not all lvalues can be modified.

Below is an exhaustive list of all expressions in the language and their value categories. If an expression is missing, then
this is a defect in the specification and should be reported.

\subsubsection{Lvalue Expressions [expr.value.lvalue]}
The following expressions are lvalues:
\begin{itemize}
\item \emph{procedure-declaration}s \sref[expr.proc], whether named or not.
\item \emph{variable-declaration}s \sref[expr.var].
\item \emph{non-empty-block-expression}s \sref[expr.block] that are not empty and whose \emph{last-expression} is an lvalue.
\item \emph{name-expression}s \sref[expr.name] that denote a variable or parameter.
\item \emph{member-access-expression}s \sref[expr.member] whose left-hand-side operand is an lvalue of structure type.
\item \emph{unary-prefix-expression}s with operator `*` \sref[expr.unary.deref].
\item \emph{binary-expression}s with assignment operators \sref[expr.binary.assign], but \emph{not} operator `=>` \sref[expr.binary.refassign].
\item \emph{if-expression}s with an \emph{else-clause} whose body and \emph{else-clause} yield lvalues of compatible types.
\end{itemize}

\subsubsection{Rvalue Expressions [expr.value.rvalue]}
The following expressions are rvalues:
\begin{itemize}
\item \emph{invoke-expression}s, including such as return a value of reference type \sref[expr.invoke].
\item \emph{block-expression}s \sref[expr.block] that are empty or whose \emph{last-expression} is an rvalue.
\item \emph{name-expresison}s that denote a procedure or type \sref[expr.name].
\item \emph{member-access-expression}s \sref[expr.member] whose left-hand-side operand is an rvalue or not of structure type.\footnote{This
      also means that you cannot e.g. assign to the \texttt{.data} or \texttt{.size} member of a slice or rebind them.}
\item \emph{literal-expression}s, including \emph{string-literal}s \sref[expr.literal].
\item \emph{binary-expression}s \sref[expr.binary] with non-assignment operators \sref[expr.binary.assign] or operator `=>` \sref[expr.binary.refassign].
\item \emph{type}s \sref[expr.type].
\item \emph{if-expression}s with no \emph{else-clause} or whose body or \emph{else-clause} yield an rvalue.
\end{itemize}

\subsection{Operand Value Categories [expr.value.operands]}
Every expression mandates a certain value category and type for each of its operands. Let $P^i$ denote the mandated
operands for an expression $E$, $A^i$ the actual operands; given an abf expression $e$, let $C(e)$ be the value category of $e$, and $T(e)$ the type
of $e$. The following algorithm is performed for each $A^i$:
\begin{enumerate}
\item If $T(A^i) \neq T(P^i)$, and there is an implicit conversion of $A^i$ (!) to $T(P^i)$, perform that conversion.
      If there is no such conversion, the program is ill-formed.
\item If $C(A^i) \neq C(P^i)$, and $C(A^i)$ is an lvalue, perform lvalue-to-rvalue-conversion; otherwise, the program
      is ill-formed.\footnote{Note that an rvalue of reference type would have already been converted to an lvalue of
      the element type in the previous step.}
\end{enumerate}

\Example
```[Source]
proc by_ref (i32&) {}
proc by_val (i32) {}

i32 i;
by_val i; /// OK: same type; lvalue-to-rvalue conv. from lvalue i32.
by_val 4; /// OK: const-expr conv. from rvalue int to i32.
by_ref i; /// OK: conv. from lvalue i32 to rvalue i32&.
by_ref 4; /// ERROR: No conv. from rvalue int to i32&.
```

\subsubsection{Operand Value Constraints by Expression Kind [expr.value.constraints]}
If an expression is not listed below, it either has no operands (e.g. \sref[expr.unreachable]) or no constraints
associated with its operands in the general case (e.g. \sref[expr.block]).

\begin{itemize}
\item For an \emph{invoke-expression}, operand type constraints are determined via overload resolution \sref[expr.overload].
      All operands must be rvalues.
\item For a \emph{variable-declaration}, if the declaration has an initialiser, the type of the initialiser must be
      an rvalue of the same type as the type of the declared variable.
\item For a \emph{member-access-expression}, the LHS operand must be of slice, range, or structure type or of
      a pointer-like type thereto.
\item For an \emph{if-expression}, the condition must be an rvalue of type `bool`.
\item For a \emph{unary-prefix-expression},
      \begin{itemize}
        \item with operator `*`, the operand must be of a pointer-like type.
      \end{itemize}
\item For a \emph{binary-expression},
      \begin{itemize}
          \item with an arithmetic operator, the operands must be rvalues of integral type;
          \item with a comparison operator, the operands must be rvalues of type `bool`;
          \item with a value-assignment operator (e.g. `=`), the LHS operand must be an lvalue
                of the same type as the RHS. The RHS must be an rvalue;
          \item with the reference-assignment operator `=>`, the LHS must be an lvalue of reference type
                and the RHS must be an rvalue of the same type. \sref[expr.binary.refassign]
      \end{itemize}
\item For a procedure body introduced with `=`, the type of the body must be an rvalue of the same
      type as the procedure’s return type.
\end{itemize}

\subsection{Binary Expressions [expr.binary]}
Binary expressions have a LHS and a RHS that are combined with an operator. The semantics of
a binary expression depend on the operator and sometimes on the types and value categories of
the operands.

\subsubsection{Reference Reassignment [expr.binary.refassign]}
The reference reassignment operator `=>` is a special binary operator that is used to rebind references. Unlike
the value assignment operator `=`, there are no compound variants of this operator. The yield of the expression
is an lvalue to the reassigned reference. The LHS of the expression must be convertible to an lvalue of reference
type $T$, and the RHS to $T$.

The interaction of this expression with value categories and autodereferencing is a bit complicated.
To elaborate, we first define the notion of the reference depth of a type. Let $\mathit{Elem}(t)$ be the
element type of a reference type $t$. The reference depth $D(t)$ of a type $t$ is $0$ if $t$ is not a reference
type, and $1 + D(\mathit{Elem}(t))$ otherwise.

The algorithm below illustrates the exact semantics of this operation. Let $l$ be the LHS and $r$ the RHS. Let $T(e)$
be the type of an expression $e$. For brevity, we may write $D(e)$ for $D(T(e))$ if $e$ is defined to be an expression.
\begin{enumerate}
\item If $D(l)$ is $0$, the program is ill-formed.
\item If $D(l)$ is $1$, and $l$ is an rvalue, the program is ill-formed.
\item If $l$ is an rvalue, perform reference-to-lvalue conversion on $l$, yielding a new $l$.\footnote{The reference
      depth of $l$ in that case is reduced by one, e.g. rvalue \texttt{int\&\&} becomes lvalue \texttt{int\&}.}
\item If $D(l) < D(r)$, perform lvalue-to-rvalue conversion on $r$, yielding a new $r$; then,
      perform dereferencing followed by lvalue-to-rvalue conversion $D(r) - D(l)$ many times, yielding a new $r$. Go
      to step 6.\footnote{This removes that many levels of indirection from the RHS, yielding an rvalue.}
\item Otherwise, if $D(l) > D(r)$, perform lvalue-to-rvalue conversion followed by dereferencing $D(l) - D(r)$ many
      times on $l$, yielding a new $l$.\footnote{This removes that many levels of indirection from the
      RHS, yielding an \emph{lvalue}, unlike in the previous step.}
\item If there is no implicit conversion from $T(r)$ to $T(l)$, the program is ill-formed.
\item Otherwise, perform that conversion, followed by lvalue-to-rvalue conversion, yielding $y$, and store $y$
      in the lvalue $l$; the result of this operation is the lvalue $l$. Skip all remaining steps.
\end{enumerate}

\subsection{Names [expr.name]}
\subsubsection{Definitions [expr.name.def]}
An \emph{unqualified-name} is an `<identifier>` token that is not part of a qualified name \sref[expr.name.def].

A \emph{qualified-name} is a sequence of one or more unqualified names separated by `.` tokens. Note that identifiers
separated by the metaproperty access operator `::` do not constitute qualified names. E.g. in the sequence
```[Source]
foo.bar::baz.quux
```

there are two qualified names: `foo.bar` and `baz.quux`. Thus, this is parsed as:
```[Source]
(foo.bar)::(baz.quux)
```

Note: `<identifier>` tokens created by the `__id` keyword are never qualified names, even if they contain `.` tokens.

\subsection{Name Lookup [expr.name.lookup]}
If the name is not a \emph{qualified-name}, unqualified name lookup is performed. Otherwise, qualified name lookup is performed.
The algorithm for unqualified name lookup is as follows:
\begin{enumerate}
\item Let `n` be the text value of the \emph{unqualified-name}; let `ref` be null. If, during the execution of any of the
      steps below, the value of `ref` is changed,
    \begin{enumerate}
        \item in a context where shadowing is enabled [expr.name.shadow], then, if the current step is a loop (‘For each’), run
              the current iteration of that loop to completion; then, skip to the last step of this algorithm;
        \item in a context where shadowing is disabled, if `ref` is already non-null, the program is ill-formed.
    \end{enumerate}
\item If we are currently analysing an \emph{enum-declaration}, let `d` be that declaration. If `d` contains an
      enumerator `e` whose name matches `n`, set `ref` to an \emph{enumerator-reference} to `e`. If `e` is not
      yet complete \sref[expr.enum], the program is ill-formed.
\item If there is an expected type \sref[expr.expected], let `t` be that type.
    \begin{enumerate}
        \item If `t` is an enum type containing an enumerator `e` whose name matches `n`, replace the name with an
              \emph{enumerator-reference} to `e`. Note that it is irrelevant whether `e` is complete at this time or not.
        \item If `t` is a record type containing a variant clause `v` whose name matches `n`, replace the name with a
              \emph{type-reference} to `v`.
      \end{enumerate}
\item For each scope in the scope tree starting at the scope containing the name, and ending at the global scope:
    \begin{enumerate}
        \item If the scope contains a declaration `d` whose name matches `n`, replace the name with a reference to `d`.
        \item If the scope contains an overload set whose name matches `n`, replace the name with a reference to that
              overload set.
        \item If a \emph{with-expression} `w` is in scope and complete [expr.with], and `w` contains a member whose name
              matches `n`, replace the name with a \emph{member-access} from `w` to `n`.
    \end{enumerate}
\item If template instantiation is currently taking place [expr.template.inst], for each template `t` on the instantiation
      stack, if there is a template parameter of `t` that matches `n`, replace the name with a copy of the template argument
      value bound to the parameter in the instantiation.
\item If `n` is the name of an imported module, replace it with a \emph{module-reference} to that module.
\item If the translation unit is a logical module [module.def] and its name is `n`, replace the name with a
      \emph{module-reference} to that module.
\item For each open imported module [module.def], check if that module exports a declaration whose name matches `n`. If so,
      replace the name with a reference to that declaration.
\item Replace the name with `ref`. If `ref` is null, the program is ill-formed.
\end{enumerate}

\noindent The algorithm for qualified name lookup is as follows:
\begin{enumerate}
\item Let `n` be the text value of the \emph{qualified-name}.
\item If there is an imported module whose name matches `n`, replace the \emph{qualified-name} with a \emph{module-reference} to that module and stop.
\item Otherwise, split `n` at the last occurrence of `.` into `n1` and `n2`. Construct a \emph{member-access} from `n1` to `n2` and
      perform name lookup on `n1` and `n2`.
\item Perform member lookup \sref[expr.member-access] on the result of the previous step.
\end{enumerate}

\subsection{Type Conversion [expr.convert]}
During semantic analysis, we often have to convert expressions from one type to another; the algorithm below
describes how this conversion is performed. This algorithm, named \alg{Convert}, both checks if the conversion
is possible and replaces the expression to be converted with either a \emph{cast-expression} or the converted value.

In some cases, we only want to check if a conversion is possible, but not actually perform the conversion. This can
be accomplished by skipping any parts of steps annotated with square brackets. This variation of the algorithm is called
\alg{Try-Convert}.

The algorithm returns an integer \emph{score} that indicates how good the conversion is, with `0` being optimal,
a negative value indicating failure, and a higher positive value indicating a higher cost. This score is used for
overload resolution \sref[expr.overload].

\begin{itemize}
    \item Let $f$ (‘from’) be the type of the expression $E$ to be converted, and $t$ (‘to’) the type to convert to.
    \item If $f = t$, return $0$.
    \item If $t$ is `void`, return $0$.
    \item If $f$ is `noreturn`, [replace $E$ with a \emph{cast-expression} of $E$ to $t$ and] return $0$.
    \item If $t$ is `type` and $E$ is a type, evaluate $E$ as a constant expression, yielding $R$; [replace $E$ with $R$ and]
          return $0$.
    \item If $E$ is an lvalue, and $t$ is a reference type whose element type is $f$, [replace $E$ with a \emph{cast-expression}
          of $E$ to $t$ and] return $0$.
    \item If $f$ and $t$ are weak pointer-like types of the same kind, then,
    \begin{itemize}
        \item if $t$ is a void pointer, [replace $E$ with a \emph{cast-expression} of $E$ to its corresponding void pointer type and] return $1$; or
        \item if the element type of $f$ is an array type whose element type is equal to the element type of $t$,
              [replace $E$ with a \emph{cast-expression} of $E$ to $t$ and] return $1$ (i.\,e. a pointer or reference to an
              array is implicitly convertible to a pointer or reference to its first element); or
        \item otherwise, return $-1$.
    \end{itemize}
    \item If $f$ is a reference type and $t$ a pointer type, and the element types of $f$ and $t$ are the same, [replace $E$ with a
          \emph{cast-expression} of $E$ to $t$ and] return $1$.
    \item If $f$ and $t$ are integer types, then,
    \begin{itemize}
        \item if $f$ is the integer literal type, and $t$ is `isz`, [set the type of $E$ to `isz` and] return $0$; or
        \item if $E$ is a constant expression, evaluate $E$ as a constant expression, yielding $R$; if the value of $R$
              is representable by $t$, [replace $E$ with $R$ and] return $1$; otherwise, return $-1$; or,
        \item if $E$ is not a constant expression, and $f$ is narrower than $t$, then—unless $f$ is signed and $t$ isn’t—[replace
              $E$ with a \emph{cast-expression} of $E$ to $t$ and] return $1$; otherwise, return $-1$; or,
        \item otherwise, return $-1$.
    \end{itemize}
    \item If either type is a slice with element type $e$, and the other is a tuple type containing two elements, namely
          pointer to $e$ and `isz`, [replace $E$ with a \emph{cast-expression} of $E$ to $t$] return $1$.
    \item If either type is a tuple type and the other a structure type that is not packed and contains no variants, and the
          tuple contains the same number of elements as the structure has fields, and each element of the tuple is convertible
          to the corresponding field of the structure, and all fields of the structure have their natural alignment, [replace
          $E$ with a \emph{cast-expression} of $E$ to $t$] return $1$.
    \item If $f$ and $t$ are integer ranges, and $E$ is a constant expression, evaluate $E$ as a constant expression, yielding $R$;
          if the start and end value of $R$ is representable by $t$, [replace $E$ with $R$ and] return $1$; otherwise, return $-1$.
    \item If $f$ is a function type and $t$ a closure type whose procedure type is equal to $f$, [replace $E$ with a \emph{cast-expression}
          of $E$ to $t$] return $1$.
    \item Otherwise, return $-1$.
\end{itemize}

\subsection{Overload Resolution [expr.overload]}
To resolve an overloaded callee of a \emph{call-expression} as well as any arguments of the call that are
also overload sets, the following algorithm is used. In the algorithm below, upper indices are associated
with candidates of overload sets, and lower indices with parameters and arguments.

Let $O$ be the overload set containing candidates $C^i$, each with parameters $C^i_j$; let $A_j$ be the
arguments of the call (note: the arguments may be unresolved overloads themselves). Let $\|C^i\|$ be the
number of parameters of $C^i$ and $\|A\|$ the number of arguments of the call expression. For each
candidate $C^i$, let $A^i_j \coloneqq \{ A_j : j \leq \|C^i\| \}$ be the \textit{non-variadic} arguments for $C^i$. Then:

\begin{enumerate}
\item Mark all $C^i$ as viable.
\item For each candidate $C^i$:
    \begin{enumerate}
        \item If $\|C^i\| > \|A\|$, i.\,e. the function takes more parameters than there are arguments, mark $C^i$ as
              non-viable and continue with the next candidate.
        \item If $\|C^i\| < \|A\|$ and $C^i$ is not variadic, mark $C^i$ as non-viable and continue with the next candidate.
        \item For each $A^i_j$ that is not an overload set, let $s^i_j \coloneqq \text{\scshape Try-Convert}(A^i_j, C^i_j)$
              as defined in [expr.convert]. If $s^i_j = -1$, mark $C^i$ as non-viable and continue with the next candidate.
        \item For each $A^i_j$ that is an overload set:
            \begin{enumerate}
                \item Let $\Gamma^k$ be the candidates of $A^i_j$.
                \item For each $\Gamma^k$, let $\sigma^k \coloneqq \text{\scshape Try-Convert}(\Gamma^k, C^i_j)$.
                \item If all $\sigma^k = -1$, mark $C^i$ as non-viable and continue with the next candidate.
                \item Otherwise, let $s^i_j \coloneqq \min \{ \sigma^k : \sigma^k \neq -1 \}$, and let $\gamma^i_j$ be the
                      candidate $\Gamma^k$ with $\sigma^k = s^i_j$. If there are multiple possible $\gamma^i_j$,
                      mark $C^i$ as non-viable and continue with the next candidate.
            \end{enumerate}
        \item Let $s^i \coloneqq \sum_j s^i_j$.
    \end{enumerate}
\item If there is no viable $C^i$, the program is ill-formed.
\item Let $\mu$ be the index such that $s^\mu = \min \{ s^i \}$ and $C^\mu$ is viable. If there are multiple possible $\mu$, the program is
      ill-formed.
\item For all $A_k$ with $\|C^\mu\| < k \leq \|A\|$ that are overload sets:
    \begin{enumerate}
        \item If $A_k$ contains more than one candidate, the program is ill-formed.
        \item Otherwise, resolve $A_k$ to its only candidate.
    \end{enumerate}
\item Resolve $O$ to $C^\mu$
\item Resolve each $A^\mu_j$ that is an overload set to its corresponding $\gamma^\mu_j$ as determined in step 2(d)iv.
\end{enumerate}

\subsection{Declarations [expr.decl]}
\subsubsection{Decl-Types [expr.decl.type]}
Not every type can occur in a declaration. Thus, in order to make sure that a type is valid, \emph{decl-type decay}
is applied to any type to be used in a declaration. The decay is defined as follows:
\begin{itemize}
\item If the type is a function type, the decayed type is the corresponding closure type.
\item Otherwise, the decayed type is the type itself.
\end{itemize}

\subsection{Scope [expr.scope]}
A new scope is opened by
\begin{itemize}
\item the global scope;
\item a block-expr that is not a delim-expr;
\item a static-if-expr;
\item an if-expr, encompassing the condition and its body (but not any elif or else branches);
\item an elif-expr, encompassing the condition and its body (but not any other elif or else branches);
\item an else-expr;
\item a while-expr, encompassing the condition and its body;
\item a for-expr, encompassing its initialisers and its body;
\item a proc-def-expr, encompassing its parameter declarations and its body;
\item a cond-expr, encompassing the standard of comparison and the body of each branch;
\item a cond-case, encompassing its pattern and its body;
\item a with-expr that has a body;
\item a defer-expr.
\end{itemize}

\noindent Note that scopes are mainly relevant for three things:
\begin{enumerate}
\item The visibility of declarations. A declaration is visible in the scope it is declared in and all scopes nested within it.
      Declarations inside of expressions that do not open a scope ‘float up’ to the nearest enclosing scope.
\item Insertion of destructor calls for local variables and parameters. Destructors are called in the reverse order of
      construction at the end of the scope they are declared in.
\item Placement of defer-exprs. The rules for the eventual placement of defer-expressions are the same as for destructors. Note
      that defer-expressions ‘float up’ as well, but since their type is `void`, they are automatically disallowed inside of
      most expressions where they would otherwise be difficult to handle, such as assert-exprs, the condition of an if-expr,
      or on either side of the `and`/`or` operators.
\end{enumerate}

\end{document}